\documentclass{article}
\usepackage{graphicx} % Required for inserting images
\usepackage{amssymb}
\usepackage{amsmath}
\usepackage{revsymb} 
\usepackage{gensymb}

\title{Simulation of Rubidium 87 BEC with Machine-learning-accelerated}
\author{Inho Choi}
\date{\today}

\begin{document}

\maketitle
\abstract{}
Machine learning is emerging as a technology that can enhance physics experiment execution and data analysis. Here, we apply machine learning to accelerate the production of a Bose-Einstein condensate (BEC) of $^{87}Rb$ atoms by Bayesian optimization of up to 55 control parameters. This approach enables us to prepare BECs of $2.8 \times 10^3$ optically trapped $^{87}Rb$ atoms from a room-temperature gas in $575 ms$. The algorithm achieves the fast BEC preparation by applying highly efficient Raman cooling to near quantum degeneracy, followed by a brief final evaporation. We anticipate that many other physics experiments with complex nonlinear system dynamics can be significantly enhanced by a similar machine-learning approach.

\section{Introduction}

Recently, researchers have begun applying machine-learning techniques to atomic physics experiments, e.g., to enhance data processing for imaging, determine the ground state and dynamics of many-body systems, or to identify phases and phase transitions. One promising practical application of machine learning to atomic physics is in the optimization of control sequences with many parameters and nonlinear dynamics, and in particular to one of the workhorses of atomic physics, Bose-Einstein condensates (BECs).

With few exceptions, experiments on BECs end with a destructive measurement, which requires repeated BEC preparation. Approaches to increase the BEC production rate, and associated signal-to-noise ratio of the experiments, have generally relied heavily on hardware improvements or have used atomic species with narrower optical transitions than offered by the most widely utilized alkali-metal atoms. For alkali-metal atoms, the tight confinement of atom-chip magnetic traps has enabled fast evaporation sequences, with a complex multi-layer atom-chip achieving BEC preparation times of $850 ms$ for $4 \times 10^4$ atoms. Nonalkali- metal atoms featuring narrow optical transitions can be used to reach lower temperatures in narrow-line magnet-ooptical traps (MOTs). That approach, combined with a dynamically tunable optical dipole trap, has recently been used to prepare BECs of $2 \times 10^4$ erbium atoms in under $700 ms$.

In this article, we demonstrate a complementary approach where, in a simple experimental setup with a broad-line MOT for a standard alkali-metal atom, machine learning is leveraged to optimize a complex nonlinear laser and evaporative cooling process to quantum degeneracy. Controlling a sequence with up to $55$ interdependent experimental parameters, Bayesian optimization finds parameter values which cool a gas from room temperature into the quantum degenerate regime in $575 ms$, creating a BEC containing $N_{BEC} = 2.8 \times 10^3$ atoms.

We identify some of the physical strategies discovered by the algorithm and also investigate how the choice of cost function impacts the trade-off between final atom number and the purity of the created BEC.

Our apparatus employs only a single MOT directly loaded from a $^{87}Rb$ background vapor, a crossed optical dipole trap, and two Raman cooling beams as depicted in Fig. 1(a). No Zeeman slower, two-dimensional MOT, atom chip, dynamic trap shaping, or strobing are necessary. Using Raman cooling in a crossed optical dipole trap (cODT), a method that can reach very high phase-space density and even condensation, the algorithm achieves a cooling slope of $16$ orders of magnitude improvement in phase space density (PSD) per order of magnitude in atom loss ($\gamma = 16$) up to the threshold to quantum degeneracy. This is significantly better than the $\gamma = 7$ value we could obtain with extensive manual optimization under similar conditions.

\subsection{Description of Fig. 1}

(a) Setup showing $1064$-nm horizontal (waist $\omega_h = 18 \mu m$, beam slightly tilted downward) and vertical ($\omega_v = 14 \mu m$) optical-trapping, $795$-nm Raman coupling ($\omega_R = 500 \mu m$) and optical pumping ($\omega_x = 30 \mu m$, $\omega_y \approx 1 mm$), and $780$-nm absorption-imaging beams. 

(b) Absorption image used to extract the cost function for a set of parameter values $\mathbf{X}$. 

(c) Bayesian optimization with a neural network. The model $C_p(\mathbf{X})$ (orange solid line) attempts to predict the actual system performance $C(\mathbf{X})$ (blue dashed line). The algorithm uses the model to predict optimal parameter values $\mathbf{X}_{i+1}$ (open diamond), tests those values, and performs a new iteration with an updated model.

\section{Atomic Physics Methods}
The Raman cooling implementation used in this work is similar to that of descripbed at Chapter \ref{sec:Raman Cooling}. Cooling proceeds in a cODT formed by intersecting two non-interfering $1064$-nm beams, one horizontal and one vertical [see Fig. 1(a)]. Two $795$-nm beams drive the Raman cooling: the optical pumping beam and the Raman coupling beam. Raman cooling provides sub-Doppler cooling by driving velocity-selective Raman transitions between hyperfine states, here the $\vert F = 2, m_F = -2 \rangle$ and $\vert 2,-1 \rangle$ states of $^{87}Rb$. The Raman transitions are nondissipative so entropy is removed from the atomic gas in the form of spontaneously scattered photons as atoms are optically pumped back to the dark state $\vert 2,-2\rangle$. Light-assisted collisions, which typically prohibit laser cooling at high atomic densities, are suppressed by detuning the optical pumping light $4.33$ GHz to the red of the $D_1 F = 2 \rightarrow F' = 2'$ transition, where a local minimum of light-induced loss was observed.

The cooling dynamics are controlled via five actuators: (i) the horizontal $P_y$ and (ii) vertical $P_z$ trap beam powers which set the trap depth and frequencies, (iii) the Raman coupling beam power $P_R$ which tunes the Raman rate, (iv) the power $P_p$ of the optical pumping beam which sets the optical-pumping rate (and also Raman rate), and (v) the magnetic field $B_z$ which adjusts the resonant velocity class for the Raman transition. The cooling procedure is divided into stages during which the controls are linearly ramped, with the endpoints of each ramp constituting the optimization parameters.

\section{Optimization Scheme}
The optimization problem can be formulated as the minimization of a cost function $\mathbf{C}$, which maps a set of parameter values $\mathbf{X} \in \mathbb{R}$ to a corresponding cost value $\mathbf{C}(\mathbf{X}) \in \mathbb{R}$, where $M$ is the number of optimization parameters. The cost $\mathbf{C}$ quantifies the results and is generally a priori unknown, but can be extracted from measurements. Bayesian optimization is well-suited for this type of problem as it can tolerate noise in the measured cost and typically requires testing fewer values of $\mathbf{X}$ than other optimization methods.

Bayesian optimization begins with collecting a training dataset by experimentally measuring the cost $\mathbf{C}_m(\mathbf{X}_i)$ for various values of sets of parameter values $\mathbf{X}_i$. The $\mathbf{X}_i$ values used to construct the training dataset are chosen by a training algorithm, which can implement another optimization algorithm or can select $\mathbf{X}_i$ randomly. A model of the cost function is then fit to the training dataset which approximates the unknown true cost function $\mathbf{C}(\mathbf{X})$. Although Bayesian optimization typically uses a Gaussian process for its model, the present work uses neural networks, which were chosen for their significantly faster fitting time for our typical number of optimization parameters. Once the model is fit, a standard numerical optimization algorithm is applied to the modeled cost function $\mathbf{C}_p(\mathbf{X})$ to determine which value $\mathbf{X}_{i+1}$ for the next iteration is predicted to yield the minimal cost, as depicted in Fig. 1(c). Optionally this numerical optimization can be constrained to a trust region (a smaller volume of parameter space centered around the $\mathbf{X}_i$ which yielded the best cost measured thus far). The predicted optimal value $\mathbf{X}_{i+1}$ is then tested by experimentally measuring the corresponding cost $\mathbf{C}_m(\mathbf{X}_{i+1})$. The next iteration begins by retraining the model with the new result, and making a new prediction for the optimal value of $\mathbf{X}$ with the updated model. The algorithm iterates until it reaches a termination criterion, such as a set maximum number of iterations or a set number of consecutive iterations that fail to return better results. All optimization in this work was performed with the open-source packages M-LOOP to implement the Bayesian optimization and LABSCRIPT for experimental control. Additional implementation details are included in Chapter \ref{sec:bayesian}.

\section{Cost Function}
Since the optimization transitions the gas from the classical into the quantum degenerate regime, the final state of the gas depends strongly on how the cost function is chosen as a combination of the two experimentally accessible parameters: atom number $N$ and temperature $T$ . The classical phase space density $\mathbf{PSD}_c$ is defined as $\mathbf{PSD}_c \equiv n_{cp} \lambda_{dB}^3$, where $\lambda_{dB}$ is the thermal de Broglie wavelength and $n_{cp}$ is the calculated peak number density neglecting bosonic statistics (see Chapter \ref{sec:atomic gas properties} for calculation details). The value of $\mathbf{PSD}_c$ is nearly equal to the true PSD when $\mathbf{PSD} \ll 1$, while at the threshold to condensation $\mathbf{PSD}_c \sim 1$. Since the temperature $T$ is more difficult to determine in the quantum degenerate regime, and also requires a fit to the data with potential convergence problems, we instead measure $N$ and the peak optical depth ($\mathbf{OD}$) in an absorption image. Generally ensembles with larger $\mathbf{PSD}_c$ have a larger atom number $N$ and less expansion energy, which leads to a larger peak $\mathbf{OD}$ for a given $N$. Guided by this, we explored cost functions of the form

\begin{equation}
    \mathbf{C}(\mathbf{X}) \propto -f(N/N_1) \mathbf{OD}^3 N^{\alpha - 9/5},
\end{equation}
where $f(N/N_1)$ is a smoothed Heaviside step function with $N_1$ chosen near the detection noise floor (see Chapter \ref{sec:bayesian}). The parameter $\alpha$ in the cost function tunes the trade-off between optimizing for larger atom number or lower temperature. For a pure BEC after sufficient time-of-flight (TOF) expansion, $|C/f|$ scales as $(N_{BEC})^\alpha$ (see Chapter \ref{sec: cost scaling}). For a thermal cloud, $|C/ f |$ is proportional to $\mathbf{PSD}_c$ when $\alpha = -1/5$, although that value of $\alpha$ is unsuitable for condensation as increasing the atom number in the BEC requires $\alpha > 0$.

\section{Optimization Procedure}
The sequence begins with a separately optimized $99$-ms long MOT loading and compression period. The trap beam powers are ramped to their initial Raman cooling values during the last $10$ ms of the MOT compression and then the magnetic field is adjusted to its initial Raman cooling value in $1$ ms, at which point the horizontal dipole trap holds typically $N = 2.7 \times 10^5$ atoms. We then added $100$-ms stages of Raman cooling one by one and optimized them individually. After five stages (Chapter \ref{sec:Raman Cooling}), the algorithm tended to turn off the Raman cooling by turning down $P_p$ or $P_R$ or by tuning the magnetic field $B_z$ such that the Raman transition became off-resonant. We then added up to six shorter $30$-ms-long stages in which the optical pumping and Raman coupling beams were turned off, and the algorithm performed evaporative cooling. Due to the reduced number of parameters, we were able to optimize the evaporation stages simultaneously, which produced a BEC. Subsequently we shortened the Raman cooling and evaporation stages with parameter values fixed until only a small and impure BEC was produced, and then we ran a global reoptimization. In this global optimization stage, all $42$ of the Raman cooling and evaporation parameters were reoptimized simultaneously using the previous values as the initial guess for $\mathbf{X}$. Often a trust region set to one-tenth of the allowed range for each parameter was used. This kept the optimizer focused in regions of parameter space which produced a measurable signal, as adjusting even a single parameter too far would often result in the loss of all atoms.We repeated this sequence shortening and reoptimization procedure until the algorithm failed to find parameters that could produce sufficiently pure BECs.

The required beam powers generally varied over several orders of magnitude, so the logarithms of their powers were used as entries in $\mathbf{X}$, while the magnetic-field control parameter $B_z$ was kept a linear parameter. A feedforward adjustment was included in the $B_z$ control values to account for the light shift of the $\vert 2,-1\rangle$ state by the optical pumping beam. We averaged over five repetitions of the experiment for each set of parameter values tested. The number of iterations per optimization varied but was typically $\sim 1000$ (including the initial training) and required several hours, both for the single-stage optimizations and the full-sequence optimizations. A simpler optimization procedure was also attempted which did not involve optimizations of individual stages. Instead the sequence was divided into ten $100$-ms stages and all $55$ parameters were optimized from scratch simultaneously. That approach combined with the shortening and reoptimizing procedure successfully produced a similar BEC, albeit in slightly longer time ($650$ vs $575$ ms), possibly due to the optimization becoming trapped in a local optimum (see Chapter \ref{sec:bayesian} for further discussion).

\section{Results and Physical Interpretation}
The best discovered $575$-ms-long control sequence and corresponding results are depicted in Figs. 2 and 3. Notably, the algorithm discovered gray molasses in the MOT phase, which it applies at the end of the compression sequence. This outperforms the bright molasses that was previously used in the manually optimized compression sequence, with the gray molasses loading a similar number of atoms ten times faster. After the MOT loading stage and transfer into the cODT, five $\sim 63$-ms-long stages of Raman cooling follow, and then the optical pumping and Raman beams are ramped off, followed by six $\sim 27$-ms-long evaporation stages. As observed in previous work, the ramps produced by Bayesian optimization are non-monotonic and appear non-intuitive, but they outperform the routines we found by manual optimization. A reason for the non-monotonic wave forms may be that the cost function includes many local minima. The optimization can settle into any one of these local optima randomly and produce complex but specific wave forms. Despite the non-monotonic ramps, $\mathbf{PSD}_c$ increases smoothly exponentially during this part of the sequence [Fig. 2(e)], due in part to the finite thermalization rate.

By shortening the sequence we are asking the algorithm to maximize the cooling speed, which is limited by the lower of the collisional rate $\nu_c$ and the trap vibration frequencies $\nu_{x,y,z}$. When the gas is still hot, we have $\nu_c \ll \nu_{x,y,z}$, and the algorithm employs Raman cooling to increase the density and collision rate [Fig. 2(d)]. However, when $\nu_c$ approaches the lowest trap vibration frequency $\nu_y$ near the time $t = 225$ ms, the algorithm starts to reduce the Raman rate, and a little later the optical pumping rate, in order to reduce light-induced collisions that scale with $\nu_c$, rather than the trap vibration frequency. Subsequently, for times $t > 225$ ms, the cooling proceeds near optimally, with the collision rate close to, but a little below, the trap vibration frequencies. Furthermore, as the system approaches condensation near $t = 410$ ms, the collision rate is somewhat lowered to reduce light-induced atom loss [Fig. 2(c)].

Another effect limiting the cooling speed is the loading of the atoms from the single horizontal trap, in which the sample is initially prepared, into the crossed dipole trap. Initially, the vertical-beam power $P_z$ is held low to avoid creating a high-density dimple region which would lead to excess loss during Raman cooling. Later, $P_z$ is ramped up to gather atoms from the horizontal trap beam into the overlap region of the cODT in order to increase the collision rate and speed up evaporative cooling. The relatively sudden ramping of the trap power up and then back down visible in Fig. 2(a) likely involves an optimal-control- like process since the trap compression and relaxation are faster than the axial period of the horizontal trap of $\sim 200$ ms.

The optimization tended to turn off the Raman cooling after five stages because the cloud temperature $T$ was below the effective recoil temperature (Chapter \ref{sec:Raman Cooling}) where Raman cooling, even with optimal parameters, becomes too slow, while leading to trap loss and heating due to light-assisted collisions. The Bayesian optimization recognized this and shut down the Raman cooling at this point, with the atomic gas close to condensation. Subsequently, at higher compression, which is primarily achieved by increasing the vertical beam power, the horizontal trap power is reduced and atoms are efficiently evaporated along the direction of gravity in the tilted potential. Note also that, once the atoms have been loaded into the crossedtrap region (after $t = 350$ ms), the algorithm makes all trap vibration frequencies similar, which provides the fastest overall thermalization and hence the fastest cooling speed.

The BEC is fully prepared at the end of the evaporation stages, $575$ ms after the start of the MOT loading. The final cloud contains $3.7 \times 10^3$ total atoms and is shown in Fig. 3(b). A bimodal fit of the cloud indicates that $2.8 \times 10^3$ atoms ($76\%$) are in the BEC. Although the sequence was optimized for speed rather than efficiency, the initial cooling occurs with a logarithmic slope $\gamma = d(\log \mathbf{PSD}_c )/d(\log N) \approx 16$.

\subsection{Description of Fig. 2}
Control wave forms [panels (a) and (b)] and measured trap and atomic-gas properties [panels (c)–(e)] of the optimized sequence. Gray, blue, and orange shadings mark the MOT loading, Raman cooling, and evaporation periods, respectively. The Raman beam power has been multiplied by $10^3$ for better visibility. $\nu_x$ , $\nu_y$, $\nu_z$ , and $\nu_h$ are the trap vibration frequencies in the x, y, and z directions and in the horizontal trap, respectively; $\nu_c$ is the atomic collision rate. $\mathbf{PSD}_c$ does not account for bosonic statistics and changes slowly while the BEC forms quickly above threshold. Calculations assume thermal equilibrium.

\subsection{Description of Fig. 3}
Results of the $575$-ms optimized sequence. 

(a) $\mathbf{PSD}_c$ vs atom number $N$. Initial cooling until $\mathbf{PSD}_c \approx 10^{-1}$ is very efficient with $\gamma \sim 16$ (gray line). The performance of the much slower (3-s-long) manually optimized sequence of Chapter \ref{sec:Raman Cooling} is shown for comparison ($\gamma \sim 7$). 

(b) Cross section of $24$-ms TOF image (inset) shows a BEC (orange fit) with small thermal wings.

\section{Cost Function Impact}
The atomic gases produced by sequences optimized for different values of $\alpha$ are presented in Fig. 4, as well as the results when optimizing for total atom number ($N$). Larger values of $\alpha$ result in more atoms, but at higher temperature and lower condensate fraction, while smaller values of $\alpha$ produce purer BECs, but with fewer atoms overall. Setting $\alpha$ to 0.5 was found to make a reasonable compromise (orange curve in Fig. 4); so this value was used for the final full-sequence optimization which yielded the data presented in Fig. 2.

We have demonstrated that Raman cooling with far-detuned optical-pumping light combined with a final evaporation can rapidly produce BECs with a comparatively simple apparatus, even with a standard alkali-metal atom which lacks narrow optical transitions. Bayesian optimization greatly eased the search for a short sequence to BEC, quickly discovering initially unintuitive yet high-performing sequences. Inspection of the parameters chosen by the algorithm reveals several physical strategies, such as adjusting a collision rate close to, but below, the trap vibration frequencies to maximize the thermalization and cooling speed while minimizing density-dependent atom loss, nonadiabatic loading into the crossed-trap dimple, and creating a nearly isotropic trap for efficient evaporation. In future applications, faster condensation can likely be achieved by including dynamical tuning of the trap size, while user intervention may be further reduced by factoring the sequence length into the assigned cost.

\subsection{Description of Fig. 4}
Cross sections of $24$-ms TOF images ($200$ averages) optimized for different values of the cost function parameter $\alpha$ (see main text) with 1-s-long sequences, demonstrating the trade-off between optimizing for atom number or temperature. Also plotted are the results of optimizing for atom number $N$ only. Inset: Condensate fraction $N_{BEC}/N$ vs $N$ for different $\alpha$’s.

\section{Bayesian Optimization Implementation}
\label{sec:bayesian}
In M-LOOP’s implementation of Bayesian optimization, the training algorithm used to pick parameters and generate a training dataset is also run periodically even after the training dataset is complete. In particular, once sufficient training data are acquired, three independent neural networks are trained. Each neural net is fully connected and consists of an input layer with one node for each optimization parameter, followed by five hidden layers with 64 nodes each and then an output layer with a single node. Once the training has completed, each neural network is used to generate a set of parameter values $\mathbf{X}$ which it predicts to be optimal, and each of those three $\mathbf{X}$’s is experimentally tested. Then another iteration of the training algorithm is performed and the $\mathbf{X}$ value it suggests is also tested. The results from all four of these measurements are included in the next training of the neural nets for the subsequent Bayesian optimization iteration. The additional iterations of the training algorithm are intended to encourage parameter space exploration and provide unbiased data.

In this work, the absorption images used to measure the cost function were generally taken after $1.5$ to $8$ ms of time-offlight (TOF) expansion. We averaged over five repetitions of the experiment for each set of parameter values tested, which took $\sim 10$s accounting for experimental and analysis overhead. Simply taking the largest optical depth measured in any single pixel of an absorption image as the $\mathbf{OD}$ makes it prone to noise, so the $\mathbf{OD}$ was set to the average $\mathbf{OD}$ of several pixels with the largest $\mathbf{OD}$ to reduce noise. To compare different sequences on an equal footing during optimizations, the trap beams were always ramped to a fixed power setting before releasing the atoms for TOF imaging. This final fixed ramp is only necessary during optimizations and is omitted from the sequence once the optimizations are complete. The smoothed Heaviside step function $f(N/N_1)$ included in the cost function ensures that the cost does not diverge at low $N$, while having little effect when $N$ is above the measurement noise floor. The form of $f(N/N_1)$ is inspired by the expression for the excited state population of a two-level system in thermal equilibrium and it is defined as
\begin{equation}
    f(N/N_1)= \begin{cases}
        \left( \frac{2}{e^{N_1/N}+1} \right) & N > 0, \\
        0 & N \leq 0.
    \end{cases}
\end{equation}

For many of the optimizations in this work, particularly those with tens of parameters, the cost function landscape is "sparse" in the sense that most sets of parameter values yield poor results with a signal below the measurement noise floor. Thus the actual performance for such $\mathbf{X}$ cannot be measured, and testing them provides little information to the model. This leads to large regions of parameter space where there is no measurable signal and the direction towards better values cannot be inferred. There are two notable consequences of this. First, for such optimizations it is generally necessary to provide initial values to the optimization which give a nonzero signal. Without a good starting point, the training dataset will often only include measurements dominated by noise, making it exceedingly unlikely for the Bayesian optimization to succeed. Second, for such optimizations it is generally helpful to specify a trust region. This limits the extent of excursions as the optimizer explores parameter space, making it more likely to test parameter values which yield a measurable signal. However, this does come at the cost that it makes it less likely for the optimizer to jump from one local minimum to another better minimum. We often performed the same optimization with and without a trust region in parallel. This could be done without significantly extending the duration of optimizations because the analysis for each iteration typically took longer than the time required to perform the experiment. Thus one optimization could run experiments while the other analyzed its most recent results. For optimizations with many parameters, the results with a trust region were typically as good as or better than those without. This is likely a consequence of that fact that, given the sparsity of the cost function landscape, it is unlikely for the optimizer to discover another local optimum. Thus, it is better for the optimizer to focus on modeling the region of parameter space around the local optimum rather than fruitlessly searching for another local optimum.

The sparsity of the cost landscape and necessity for providing initial parameters which produce measurable results posed a difficulty when we optimized an entire sequence from scratch at once (rather than initially adding one cooling stage at a time). We resolved this by reducing the time of flight to $1.5$ ms for the first optimization. With such a short time of flight, even poor parameter values could produce clouds with a peak optical depth above the measurement noise floor. Due to the finite dynamic range of the absorption imaging, the results of this first optimization produced a cloud which saturated the measurement and thus made it impossible to accurately quantify performance for the best-performing values. The next optimizations were performed with the same sequence duration, but the time of flight increased to $5$ ms and then to $8$ ms. This made it possible to better discern differences between high-performing sets of parameter values at the cost of increasing the performance required to produce a signal above the noise floor and thus increasing the sparsity of the cost function landscape. The procedure of shortening and then reoptimizing the sequence was then applied, resulting in the sequence presented in Fig. 5(b), which produced a BEC in $650$ ms. The parameter $\alpha$ was set to 0.5 throughout this procedure.

Although it is not strictly fair to do so due to the differing parametrizations, it is still informative to compare the control wave forms of the independently optimized 650-ms sequence to those from Fig. 4. These wave forms are presented in Fig. 5. The sequences of Fig. 4 optimized for different $\alpha$’s all had fairly similar wave forms. On the other hand, the $650$-ms sequence had a qualitatively different wave form. For example, it lacks the sudden rise and drop in vertical trap power towards the end of the sequence present in the other wave forms. This suggests that it has converged to a qualitatively different local optimum. On the other hand, the sequences of Fig. 4 were all optimized with a trust region and the same initial $\mathbf{X}$. Thus, those optimizations primarily performed a local search, only slightly tuning $\mathbf{X}$ to tailor the sequence for their particular value of $\alpha$. Although there are small differences in parametrization, the fact that two different optimizations with the same value of $\alpha$ produce sequences that differ more than optimizations with the same initial $\mathbf{X}$ but different $\alpha$’s supports the notion that the cost function landscape includes multiple local minima, as suggested in the main text.

\subsection{Description of Fig. 5}

(a) The control wave forms for the 1-s sequences corresponding to the results presented in Fig. 4, as well as the initial wave form used as the starting point for each of those optimizations. 

(b) The control wave forms for the 650-ms ten-stage sequence optimized from scratch rather than stage-by-stage initially, which was optimized with $\alpha = 0.5$. Note the differing limits for the $x$ axes between panels (a) and (b). The sequences in panel (a) include a $50$-ms magnetic coil ramp duration which was later reduced to $1$ ms. The MOT sections of the sequences have been omitted for simplicity. Note that the wave forms in panel (a) are mostly qualitatively similar despite being optimized for different cost functions. The wave forms in panel (b) are more qualitatively distinct from those in panel (a), even for the orange curves which were also optimized with $\alpha = 0.5$. This suggests that the independent optimizations likely become trapped around disparate local minima. On the other hand, tuning the cost function while providing the same initial parameter values each time typically causes only smaller deviations around the initial values.

\section{Calculations of Atomic Gas Properties}
\label{sec:atomic gas properties}

The classical phase-space density is defined as $\mathbf{PSD}_c = n_{cp}\lambda^3_{dB}$, where $n_{cp}$ is the peak number density calculated for a classical gas (i.e., neglecting Bosonic statistics) and $\lambda_{dB} = h/ \sqrt{2\pi m k_B T}$ is the de Broglie wavelength. Here $h$ is the Planck constant, $m$ is the mass of an atom, and $k_B$ is the Boltzmann constant. To calculate $\mathbf{PSD}_c$ for a cloud, its atom number $N$ and temperature $T$ are measured and it is assumed to be in thermal equilibrium. The value of $\lambda_{dB}$ is easily calculated from the measured temperature. The partition function $Z = \int f_B (\mathbf{x}) dV$ is then calculated by numerically integrating the Boltzmann factor $f_B(\mathbf{x}) = \exp[-U(\mathbf{x})/(k_BT )]$ over the trap volume, where $U(\mathbf{x})$ is the trap potential at position $\mathbf{x}$. The $U(\mathbf{x})$ is taken to be the sum of two Gaussian beams, one for each cODT beam, and gravity is neglected for simplicity. Each Gaussian beam, with peak depth $U_{i.0}$ and waist $\omega_{i,0}$, contributes a potential of the form
\begin{equation}
    U_i(\mathbf{x})=U_{i,0}\left(\frac{\omega_{i,0}}{\omega_i(z')}\right)^2 \exp \left(\frac{-2(r')^2}{\omega_i (z')^2}\right),
\end{equation}
where $\omega_i(z')=\omega_{i,0}\sqrt{1+(z'/z_R)^2}$ is the spatially varying beam width and $z_R = \pi \omega^2_{i,0}/\lambda$ is the Rayleigh range. The primed coordinates $z'$ and $r'$ are taken to be along and perpendicular to the beam’s propagation direction, respectively. The value of $n_{cp}$ can be calculated as $N f_B(\mathbf{x}_0)/Z$, where $\mathbf{x}_0$ is the position of the bottom of the trap. Finally $\mathbf{PSD}_c$ is evaluated from its definition in terms of $n_{cp}$ and $\lambda_{dB}$. Notably, for much of the sequence the atomic cloud extends out of the cODT region and into the wings of the horizontal ODT, in which case the trap potential seen by the cloud is not harmonic. Thus the well-known result $\mathbf{PSD}_c = N (\hbar \bar{\omega})^3 / (k_B T)^3$ for a harmonic trap with geometric mean trap frequency $\bar{\omega}$ cannot be used for most of the sequence.

Calculation of the mean collision rate $\nu_c$ requires averaging the collision rate $n_c \sigma \nu_{rms}$ over the cloud, where $\sigma$ is the atomic collision cross section and $\nu_{rms}$ is the root-mean-square relative velocity of atoms in the cloud. The value of $n_c$ varies over the trap and obeys $n_c(x) = N f_B(x)/Z$, again neglecting Bosonic statistics. From equipartition for a three-dimensional gas, $(1/2)\mu \nu^2_{rms} = (3/2)k_B T$ , where $\mu = m/2$ is the reduced mass for two atoms. Thus the value of $\nu_{rms}$ is given by $\sqrt{6 k_B T/m}$. The local collision rate is averaged by integrating $n_c \sigma \nu_{rms}$ over the cloud, weighted by the one-atom number density $n_c/N$, yielding
\begin{equation}
    \nu_c = N \sigma \sqrt{\frac{6 k_B T}{m}} \int\left(\frac{f_B (x)}{Z}\right)^2 dV.
\end{equation}

The above calculations assume that the cloud is in thermal equilibrium, which is often a good approximation. However, after about $440$ ms of the final optimized $575$-ms sequence, the power in the vertical trapping beam $P_z$ is rapidly increased, as can be seen in Fig. 2(a). This change is likely nonadiabatic for atoms in the wings of the horizontal ODT and the cloud may no longer be in thermal equilibrium. This is likely why the calculated $\mathbf{PSD}_c$ appears to increase beyond $\sim 1$ before the appearance of a BEC. Notably this nonadiabatic portion of the sequence occurs only after $\mathbf{PSD}_c$ has reached 0.4, and thus it does not affect the cooling efficiency estimate of $\gamma \approx 16$ for the cooling up to $\mathbf{PSD}_c = 0.1$.

The peak trap depth $U_{i,0}$ for each beam was determined from the beam waist $\omega_{i,0}$ and the radial trap frequency $\omega_{i,r}$ measured for each beam. The beam waists, defined as the radius at which the intensity falls to $1/e^2$ of its peak value, were measured by profiling the trap beams on a separate test setup which focused the light outside of the vacuum chamber. The trap frequencies were directly measured by carefully perturbing the position of a cloud in the cODT and observing its oscillations. Before perturbing the cloud, it was first cooled sufficiently to make it well confined to the central region of the cODT so that the potential was approximately harmonic. The peak trap depth for each beam could then be calculated as $U_{i,0} = m \omega^2_{i,r} \omega^2_{i,0}/4$. This expression can be derived by equating the spring constant for the trap in the radial direction $k = d^2 U_i(\mathbf{x})/(dr')^2 \vert_{\mathbf{x}=\mathbf{x}_0}$ to its value for a harmonic oscillator $k = m \omega^2_{i,x}$.

\section{Cost Scaling}
\label{sec: cost scaling}

The peak optical depth (OD) of a pure BEC after sufficient time-of-flight expansion scales as $OD \propto N_{BEC}/A$, where $A$ is the area of the cloud in the image. The area scales in proportion to $\bar{\nu}^2$, where $\bar{\nu}$ is the expansion velocity, which is related to the BEC chemical potential via $(1/2)m\bar{\nu}^2 = (2/7)\mu$ in a harmonic trap. Thus, $A \propto \mu$. Furthermore, the chemical potential for a harmonically trapped BEC scales as $\mu \propto N^{2/5}_{BEC}$, so $A \propto N^{2/5}_{BEC}$ and $OD \propto N^{3/5}_{BEC}$. The expression $OD^3 N^{\alpha-9/5}$ then scales as $(N_{BEC})^\alpha$. Notably this scaling also applies to a harmonically trapped BEC when imaged in situ. There, the BEC radius $R$ scales as $R \propto N^{1/5}_{BEC}$. In that case, $A \propto R^2 \propto N^{2/5}_{BEC}$ as before. The same arguments then apply again, indicating that $OD^3 N^{\alpha-9/5}$ scales as $(N_{BEC})^\alpha$ for a harmonically trapped BEC in situ just as it does for a BEC after a long time-of-flight expansion.

The scaling of $OD^3 N^{\alpha-9/5}$ for a purely thermal cloud is also of note. For a harmonically trapped thermal cloud, the RMS size in a given direction for any time of flight is proportional to $T^{1/2}$, so $A \propto T$ . Thus $OD \propto N/T$ and $OD^3 N^{\alpha-9/5}$ scales in proportion to $N^{\alpha+(6/5)}/T^3$. Clouds with smaller temperatures are favored by the cost function, and clouds with larger atom numbers are favored as long as $\alpha > -6/5$. For the case $\alpha = -1/5$, the value of $OD^3 N^{\alpha-9/5}$ scales in proportion to $N/T^3$, which is proportional to $\mathbf{PSD}_c$. That choice of $\alpha$ was often used when optimizing individual stages before reaching the threshold to BEC. However, note that this choice of $\alpha$ leads to the scaling $OD^3 N^{\alpha-9/5} \propto N^{-1/5}_{BEC}$ for a pure BEC and is thus not a good choice when the cloud reaches condensation.

\section{Raman Cooling Laser}
Standard Doppler cooling requires a laser with a linewidth narrow compared to the optical transition linewidth in order to achieve optimal temperatures. This places stringent technical requirements for Doppler cooling on narrow optical transitions. By contrast, Raman cooling can achieve similar velocity resolution and associated temperatures with a comparatively broad laser. In this work, the light for the Raman coupling and optical pumping beams, which drive the up-leg and down-leg of the Raman transition, respectively, was derived from the same laser. This ensures that any laser frequency noise is common mode between the two legs of the Raman transition and makes it possible to resolve Doppler shifts much smaller than the laser linewidth. A Distributed Bragg Reflector (DBR) laser diode (Photodigm PH795DBR180TS) without an external cavity was sufficient to generate the Raman cooling light. The forgiving laser linewidth requirements further simplify implementation of our BEC production approach compared to schemes which require Doppler cooling on narrow optical transitions. Thus our approach may be useful even for species which include narrow optical transitions.

\section{Direct Laser Cooling to Bose-Einstein Condensation in a Dipole Trap}
\label{sec:Raman Cooling}
\subsection{Introduction}
This section is main reference that presenting a method for producing three-dimensional Bose-Einstein condensates using only laser cooling. The phase transition to condensation is crossed with $2.5 \times 10^4$ $^{87}Rb$ atoms at a temperature of $T_c = 0.6 \mu K$ after $1.4$ s of cooling. Atoms are trapped in a crossed optical dipole trap and cooled using Raman cooling with far-off-resonant optical pumping light to reduce atom loss and heating. The achieved temperatures are well below the effective recoil temperature. We find that during the final cooling stage at atomic densities above $10^{14} cm^{-3}$, careful tuning of trap depth and optical-pumping rate is necessary to evade heating and loss mechanisms. The method may enable the fast production of quantum degenerate gases in a variety of systems including fermions.

\subsection{Main}
Quantum degenerate gases provide a flexible platform with applications ranging from quantum simulations of many-body interacting systems to precision measurements. The standard technique for achieving quantum degeneracy is laser cooling followed by evaporative cooling in magnetic  or optical traps. Evaporation is a powerful tool, but its performance depends strongly on atomic collisional properties and it requires removal of most of the initially trapped atoms. Laser cooling gases to degeneracy could alleviate those issues, but it has proven difficult to implement.

The elusiveness of laser cooling to Bose-Einstein condensation (BEC) for more than two decades can be understood as follows: optical cooling requires spontaneous photon scattering that moves entropy from the atomic system into the light field. Such scattering sets a natural atomic temperature scale $T_r$ associated with the recoil momentum from a single photon of wavelength $2 \pi \lambdabar$ and an associated atomic thermal de Broglie wavelength $\lambda_{dB} = \sqrt{2 \pi \hbar^2 / (m k_B T_r)} \sim \lambdabar$. Here $\lambdabar$ is the reduced Planck constant, $m$ the atomic mass, and $k_B$ the Boltzmann constant. BEC must then be achieved at relatively high critical atomic density $n_c \sim \lambda^{-3}_{dB} \sim \lambdabar^{-3}$, where inelastic collisions result in heating and trap loss. In particular, light-induced collisional loss becomes severe when $n \geq \lambdabar^{-3}$.

For strontium atoms, where cooling on a spectrally narrow transition is available, a strongly inhomogeneous trapping potential has been used to cool a lower-density thermal bath while decoupling the emerging condensate from the cooling light. Very recently, based on similar principles to the ones presented here, an array of small, nearly one-dimensional condensates has been prepared using degenerate Raman sideband cooling of $^{87}Rb$ atoms in a two-dimensional optical lattice.

In this chapter, we demonstrate Raman cooling of an ensemble of $^{87}Rb$ atoms into the quantum degenerate regime, without evaporative cooling. Starting with up to $1 \times 10^5$ atoms in an optical dipole trap, the transition to BEC is reached with up to $2.5 \times 10^4$ atoms within a cooling time of $\sim 1$s. As discussed in detail below, the essential components of our technique are (i) the use of carefully far-detuned cooling light to reduce atom loss and heating at high atomic densities, (ii) a reduced optical pumping rate during the final stage to avoid heating by photon reabsorption (festina lente regime), (iii) a final cooling of atoms in the high-energy wings of the thermal velocity distribution to achieve subrecoil cooling, and (iv) careful control of the final trap depth to reduce heating induced by inelastic three-body collisions.

Raman cooling of the optically trapped atoms is a two-step process where kinetic energy is first removed via a stimulated two-photon Raman transition that simultaneously changes the internal atomic state. Subsequently, entropy is removed in an optical pumping process that restores the original atomic state via the spontaneous scattering of a photon [see Figs. 6(c) and 6(d)]. The optical pumping into the state $\vert 5 S_{1/2}; F = 2, m_F = -2 \rangle$ along the $z$ axis is performed with $\sigma^{-}$-polarized light. We reduce light-induced loss by using optical pumping light with large negative detuning $\Delta / (2\pi) = -4.33 GHz$ from the $\vert 5 S_{1/2}; F = 2 \rangle$ to $\vert 5 P_{1/2}; F'=2 \rangle$ transition of the $D_1$ line, choosing a detuning far from molecular resonances [see Figs. 6(b) and 6(c) and Chapter \ref{sec: supple}. The far-detuned $\sigma^-$-polarized beam and a $\pi$-polarized beam of similar detuning which propagates in the $x-y$ plane [see Fig. 6(a)] drive the stimulated Raman transition to the state $\vert 5 S_{1/2}; 2, -1\rangle$, simultaneously changing the atomic momentum by the two-photon recoil $\hbar (\Delta \mathbf{k})$. To cool all three directions, we choose $\hbar (\Delta \mathbf{k})$ to have a nonzero projection along any trap eigenaxis.

Cooling is performed in several stages to allow optimization of the cooling as the atomic temperature and density change. Within each stage, the trapping beam powers, optical pumping rate $\Gamma_{sc}$, Raman coupling Rabi frequency $\Omega_R$, and Raman resonance detuning $\delta_R$ are held constant. The first two cooling stages, S1 and S2, are performed in a single-beam optical dipole trap (SODT), after which the atoms are transferred to a crossed optical dipole trap (XODT), where we perform three more cooling stages [X1 to X3, see Fig. 7(a)]. During each stage, we characterize the cooling performance using time-of-flight absorption imaging, extracting the atom number $N$ and temperature $T$. (For the final cooling stage close to the BEC threshhold, we exclude the central part of the time-offlight image from the temperature fit.) We quantify the cooling performance by the classical phase space density $\mathbf{PSD}_c = N[\hbar \bar{\omega} / (k_B T)]^3$, where $\hbar{\omega} = (\omega_x \omega_y \omega_z)^{1/3}$ is the geometric mean of the three trapping frequencies. Far from degeneracy, i.e., for a classical gas, $\mathbf{PSD}_c$ is equal to $n(0) \lambda_{dB}^3$, the true PSD at the center of the cloud. The parameters of each stage are optimized to yield the highest PSDc at the end of the stage [Chapter \ref{sec: supple}].

We load $1 \times 10^5$ atoms from a magneto-optical trap into the SODT propagating along the $y$ direction with a $10 \mu m$ waist [Fig. 6(a)]. After cooling in stage S1 (see Fig. 7), the trap power and vibration frequencies are reduced, thereby lowering the density and therefore the loss in stage S2. For all stages, we verify that the trap remains sufficiently deep to keep evaporative cooling negligible.

During stages S1 and S2 we perform fast cooling over $500 ms$ from $30 \mu K$ down to $1.5 \mu K$, and up to $\mathbf{PSD}_c$ just below unity. The larger loss rate during S1 relative to the other stages [see Fig. 7(c)] occurs because the Raman cooling cycle, and hence light-induced collision rate, is faster. The initial cooling at high temperatures $T \approx 30 \mu K$ and densities $n < 7 \times 10^{13} cm^{-3}$ is quite efficient, with a logarithmic slope of $\gamma = -d(\ln \mathbf{PSD}_c)/d(\ln N) = 7.2$ [see Fig. 9(a)].

After stage S2, the ensemble is sufficiently cold so that it can be efficiently transferred in the XODT by ramping up the second trapping beam ($18 \mu m$ waist, propagating along $x$) and applying a short initial cooling (stage X1). In a similar fashion to the $S1-S2$ sequence, we reduce the confinement of the XODT and cool further during X2, after which the ensemble is at the threshold to condensation. No condensate appears in X2 despite PSDc reaching the ideal-gas value of 1.2, which we attribute to a combination of the finite size effect, the interaction shift, and small calibration errors. After further reduction of the XODT, we are able to cross the BEC transition during X3, as shown by the appearance of a condensed fraction in the velocity distribution [see Fig. 9(b)]. The onset of BEC is further confirmed by the anisotropic expansion of the central part of the cloud due to trap confinement anisotropy [see Fig. 9(c)].

In order to achieve BEC, the trap depth during X3 must not be too large. For trap depths much larger than $\sim k_B \times 20 \mu K$ we observe a strong, density-dependent anomalous heating when the Raman cooling is turned off. (Heating rates of up to $10 \mu K/s$ are observed at $n \approx 2 \times 10^{14} cm^{-3}$ with a trap depth of $k_B \times 250 \mu K$.) We surmise that at high atomic densities $n \geq 10^{14} cm^{-3}$, recombination products of inelastic three-body collisions undergo grazing collisions with trapped atoms, depositing heat in the cloud. This limit on the maximum trap depth is akin to the necessity to maintain, even in the absence of evaporative cooling, a sufficiently low trap depth by applying a so-called "rf shield" in magnetic traps which allows highly energetic nonthermal atoms to escape.

The BEC transition is crossed with $N \approx 2.5 \times 10^4$ atoms at a critical temperature of $T_c = 0.61(4) \mu K$. We are able to reach condensate fractions $N_0/N$ up to $7\%$. We also verify that if the Raman cooling is turned off during stage X3, the PSD does not increase, and a condensate does not appear. Furthermore, the condensate can bemaintained for $\sim 1$s after creation if the cooling is left on, but if the cooling is turned off, the condensate decays within $\sim 100 ms$. This confirms that evaporation is insufficient to create or maintain a condensate in this trap configuration, and that the laser cooling is responsible for inducing the phase transition.

For most laser cooling methods, the requisite spontaneous photon scattering sets a recoil temperature limit. Nonetheless, we achieve subrecoil temperatures by addressing atoms in the high-energy wings of the thermal distribution. The optical pumping $\vert 2, -1 \rangle \rightarrow \vert 2, -2 \rangle$ requires on average the spontaneous scattering of three photons, and therefore imparts $6E_r$ of energy, where $E_r = \hbar^2 / (2m \lambdabar^2) = h \times 3.6 kHz$ is the recoil energy of a $795$ nm photon. As a result, only atoms with $K_{\Delta \mathbf{k}}/h \geq 29 KHz$ of kinetic energy along the $\Delta \mathbf{k}$ direction can be cooled at all [Chapter \ref{sec: supple}]. This sets an effective recoil temperature $T^{eff}_r = 2.8 \mu K$. We achieve cooling below this effective recoil limit down to $0.5 \mu K$, i.e., a mean kinetic energy $\langle K_{\Delta \mathbf{k}}\rangle / h = 5.2 kHz$, by detuning the Raman coupling so that atoms with more than the average kinetic energy are addressed by the cooling light [Chapter \ref{sec: supple}]. However, this slows down the cooling for temperatures below $T^{eff}_r$ (stage S2 onwards), while inelastic collisions add an increased heat load at high densities. In X3 when we cross $T_c$, we find that under optimized cooling the Raman transition removes $15$ kHz of kinetic energy, $30\%$ less than the expected $6E_r$ of heating (see Chapter \ref{sec: supple}). This could indicate that the cooling is aided by bosonic stimulation into the condensate during the photon scattering process.

The improved performance of our scheme compared to previous Raman cooling results is primarily due to the flexibility to perform optical pumping to a dark state at large detuning from atomic resonances by operating on the $D_1$ line. To identify suitable detunings, we separately characterized light-induced losses over a $16$ GHz frequency range to the red of the bare atomic transition, as shown in Fig. 9(a), and further detailed in the Chapter \ref{sec: supple}. Figure 9(b) displays the final atom number and condensate fraction of the optimized sequence as a function of detuning around the value of $-4.33$ GHz chosen for the experiment. We find that a suitable detuning has to be maintained within $\pm 50$ MHz to ensure good cooling performance.

Another parameter that needs to be optimized is the photon scattering rate $\Gamma_{sc}$ for optical pumping into the $\vert 2, -2 \rangle$ dark state. Despite the large detuning, the reabsorption of spontaneously scattered optical pumping photons by other atoms is a resonant two-photon process that can lead to excess heating. However, it was shown theoretically, and confirmed experimentally, that the excess heating can be suppressed at sufficiently low scattering rate $\Gamma_{sc}$, such that the confinement and two-photon Doppler broadening reduce the reabsorption probability. This limit is known as the festina lente regime. The degradation of the performance at larger $\Gamma_{sc}$ in Fig. 9(c) is consistent with increased rescattering, as the calculated reabsorption probability approaches unity. A too small value of $\Gamma_{sc}$, on the other hand, leads to higher temperatures as parasitic heating mechanisms cannot be compensated when the cooling is too slow.

While 87Rb has relatively favorable collision properties (low two-body inelastic loss rate coefficient and moderate three-body loss rate coefficient in the upper hyperfine manifold), these properties are not unique, and other atomic species may also be suitable for direct laser cooling to BEC. Since the cooling is not deeply subrecoil, relatively high densities $n \sim \lambdabar^{-3}$ are required for reaching BEC. Thanks to the fast cooling, the effect of inelastic loss is small enough if a cloud is stable at these densities (typical lifetime $\geq 1$ s at $10^{14} cm^{-3}$). Inelastic processes can be further reduced in an effectively one-dimensional geometry, where fermionization of the bosonic wave function reduces collisional processes. The demonstrated technique could also be directly applied to fermionic atoms, as well as to laser cooled molecules.

In conclusion, we have realized the decades-old goal of BEC purely by laser cooling by creating a single, moderately sized Bose-Einstein condensate in a standard crossed optical dipole trap. Notably, the method is consistent with the general theoretical recipe put forward by Santos et al. [26].

\subsection{Description of Fig. 6}
(a) Geometry of the experimental setup with $795 nm$ optical pumping and Raman coupling beams, and $1064 nm$ trapping beams. 

(b) Molecular potentials. Light-assisted collisions are suppressed if the detuning from atomic resonance $\Delta$ is chosen to be far from photoassociation resonances (solid red horizontal lines). 

(c) Partial atomic level scheme. The Raman transition is resonant for atoms with a two-photon Doppler shift $\delta_R$. 

(d) Velocity distribution of the atoms along the two-photon momentum $\hbar (\Delta \mathbf{k})$. A Raman transition reduces the velocity of atoms in the velocity class $\delta_R / | \delta \mathbf{k} |$ by $\hbar(\Delta \mathbf{k})/m$.

\subsection{Description of Fig. 7}
(a) Schematic of the trapping potential during the cooling sequence, along with the values of the trapping frequencies for each cooling stage. 

(b) Atomic temperature $T$ as a function of cooling time $t$. Discrete jumps are caused by changes of the trapping potential between the cooling stages. Inset: Temperature on a linear scale. 

(c) Atom number (open symbols) and condensate fraction $N_0/N$ (solid circles) during the cooling sequence. 

(d) Classical phase-space density $\mathbf{PSD}_c$ (see main text) as a function of cooling time $t$. The gray shaded area denotes the quantum degenerate region. Panels (b)–(d) are all plotted along the same time axis.

\subsection{Description of Fig. 8}
(a) Classical phase-space density $\mathbf{PSD}_c$ as a function of remaining atom number $N$. The cooling is very efficient until $\mathbf{PSD}_c \sim 1$ is reached. The black symbols denote the performance of the same sequence with the initial atom number reduced by a factor $5$. The final atom number is only reduced by a factor $2$, indicating a density limit in the cooling. The solid (dashed) black line indicates the $S1-S2$ $(S1-X3)$ path with an efficiency $\gamma = 7.2$ ($\gamma = 11$) for each case. 

(b) Line optical density (dots) of the cloud in time of flight along the $y'$ direction (slightly rotated from the $y$ direction in the $x-y$ plane, see Chapter \ref{sec: supple}). The data are taken after $1.6$ s of cooling and fitted with a $g_{5/2}$ Bose distribution with a Thomas-Fermi distribution superimposed (orange line). The shaded area indicates the condensed fraction. 

(c) False-color image of the optical density (OD) of the same cloud (before integration along the vertical direction), showing the anisotropic expansion of the condensed fraction in the center.

\subsection{Description of Fig. 9}
(a) Photoassociation loss spectrum. Survival probability of trapped atoms as a function of the detuning $\Delta$ of the optical pumping beam, when scattering $\sim 100$ photons. In substantial portions of the spectrum, the atomic loss is large, due to photoassociation resonances, whereas the peaks correspond to gaps in the photoassociation spectrum away from resonances. The green arrow indicates the detuning used for the data in Figs. 7 and 8. 

(b) Performance of the full cooling sequence as a function of optical pumping detuning $\Delta$ near a locally optimal detuning. A condensate fraction $N_0/N$ is visible only when the losses are limited. To keep the Raman resonance detuning $\delta_R$ constant between data points, the magnetic field is adjusted to compensate the change in light shift associated with varying $\Delta$. 

(c) Temperature and condensate fraction as a function of the scattering rate $\Gamma_{sc}$ in the final cooling stage. Here, the intensity of the $\pi -$ polarized beam is adjusted to keep the Raman coupling strength constant between data points, and $\delta_R$ is adjusted to optimize the cooling performance for each data point.

\section{Direct laser cooling to Bose-Einstein condensation in a dipole trap: Supplemental material}
\label{sec: supple}

\subsection{OPTICAL PUMPING DETUNING}
To find an optical pumping detuning that avoids losses from molecular resonances as predicted, we start with a cold cloud of atoms and measure the loss spectrum below threshold (i.e. bare atomic resonance) for light-atom detunings $\Delta$ above $-18$ GHz. This is closer to resonance than previously explored experimentally on the $D_1$-line of $^{87}Rb$. After executing a sequence which scatters the same number of photons, about $100$, regardless of detuning, we obtain a fraction of surviving atoms as given in Fig. 9(a) of the main text, with frequency regions where the remaining atom number is relatively large shown in Fig. S1. The scan is performed with a resolution of $10$ MHz. The optimum around $-4.33$ GHz in Fig. S1(b) was chosen due to its good performance. Other optima could be used as well, and better optima may exist outside the range of this scan.

\subsubsection{Description of Fig. S1}
Survival probability of the atoms in the trap as a function of detuning after repeated cycles of optical pumping, corresponding to the scattering of about 100 photons, magnified on two exemplary detuning ranges. Fig. S1(b) is centered around the value used for most of the cooling data presented in the main text.

\subsection{EXPERIMENTAL DETAILS}
The cloud is prepared in a similar way shown later. $Rb$ atoms are loaded into a MOT from a thermal vapor, followed by a compressed MOT stage where the optical pumping power is strongly reduced, such that the atoms occupy the $F = 1$ ground state manifold. The ODT is turned on at all times and about $1\times10^5$ atoms are loaded into it when the MOT fields are switched off.

The ODT propagating in the $y$-direction (see Fig. 6 in the main text), in which the atoms are originally loaded, is focused to a waist of about $10 \mu m$, while the second ODT propagating in the $x$-direction has a waist of about $18 \mu m$ at the position of the atoms. To avoid interference, the beams differ in frequency by $160$ MHz. The powers of each trapping beam throughout the sequence are shown in Fig. S2. The (calculated) total trap depths, excluding the in sequence of gravity but including counter-rotating terms, are $430 \mu K$ in stage S1 and $14 \mu K$ in stage X3. 

The imaging axis is the same as that of the $\pi$ beam, which is slightly rotated by $\approx 17 \degree$ from the $x-axis$ in the $x-y$ plane. In Fig. 8(b)-(c), we denote the horizontal axis of the image as $y'$, which is rotated from the $y$-axis in the $x-y$ plane by the same angle.

The $\sigma^-$ optical pumping beam at $795$ nm has a highly elliptical shape at the position of the atoms, with waists of $30 \mu m$ along the $x$-direction and $\sim 1 mm$ along the $y$-direction, to optimally address atoms along the sODT. The pumping beam creates a sizable light shift $\delta_{LS}$ of the $\vert 2, -1 \rangle$ state, given by $\delta_{LS}/\Gamma_{sc} = \Delta / \Gamma$, where $\Gamma$ is the natural linewidth of the $5P_{1/2}$ excited state. Since the state $\vert 2, -2 \rangle$ is dark for the $\sigma^-$ light, there is no appreciable light shift on this state. The light shift is at its largest in stage S1, reaching $\delta_{LS} / (2\pi) = 500$kHz. It is determined experimentally by measuring the shift in the Raman resonance. From the measured values of $\delta_{LS}$ we deduce the scattering rates $\Gamma_{sc}$ for $\vert 2, -1 \rangle$, shown in Fig. S2. Since the detuning $\Delta$ is comparable to the hyperfine splitting of the ground state, the same light also pumps atoms out of the $\vert 5 S_{1/2}; F=1\rangle$ manifold

The $\pi$ Raman beam propagates in the $x-y$ plane, with a waist of $0.5 mm$. Its light is derived from the same laser that generates the $\sigma^-$ beam, but it is detuned by 2 MHz from the $\sigma^-$ using acousto-optic modulators to avoid interference. This makes laser frequency noise common mode between the two beams, thereby loosening the requirements on laser linewidth necessary to have a narrow Raman transition. The Raman beam follows the same path as the light used for absorption imaging, and therefore it is circularly polarized. Over the course of the cooling only its $\pi$-polarized component (polarization along the $z$-axis, i.e. half its power), contributes to the Raman coupling. The other half of the power only adds a negligible amount of scattering and light shift to the $\vert 2, -2 \rangle$ state. Very little power ($\leq 100 \mu W$) is required to obtain significant Raman coupling (several kHz, see Fig. S2), so the light shift and scattering rate induced on the $\vert 2, -2 \rangle$ state are limited to $\geq 0.3$kHz and $\geq 2 s^{-1}$, respectively, even at the largest powers used in the cooling sequence.

For all the data presented, each data point is evaluated as an average of 3 to 5 time-of-flight absorption images.

\subsubsection{Description of Fig. S2}
Optimized multistage Raman cooling sequence. Initially we cool in single beam ODT (blue), then ramp up a second trapping beam (red) to load into a crossed ODT for the final stages of cooling (green).

\subsection{EFFECTIVE RECOIL LIMIT}
At trap frequencies in the range $0.1-5$kHz, smaller than the Raman coupling, the cooling operates mostly in the free space limit with unresolved motional sidebands and outside the Lamb-Dicke regime. Each Raman transition from $\vert 2, -2 \rangle$ to $\vert 2, -1 \rangle$ transfers $-\hbar \Delta \mathbf{k}$ of momentum to the atoms, where $\Delta \mathbf{k}$ is the difference between the wave vectors of the $\sigma^-$ and $\pi$ photons. The kinetic energy removed during one Raman transition for an atom of initial velocity $\mathbf{v}$ is:
\begin{align}
    \Delta \mathbf{K}_{Raman}&=\hbar \Delta \mathbf{k} \cdot \mathbf{v} - \frac{\hbar^2 | \Delta \mathbf{k} | ^2}{2m} \\
    &= \hbar \delta_{\mathbf{v}} - 2 E_r
\end{align}
where $\delta_\mathbf{v} = \Delta \mathbf{v} \cdot \mathbf{v}$ is the atom's two-photon Doppler shift. The orthogonality of the $\pi$ and $\sigma^-$ beams results in $\vert \Delta \mathbf{k} \vert ^2 = 2 \vert \lambdabar \vert ^{-2}$, and $E_r = \hbar^2 / (2m \lambdabar^2)$ is the recoil energy of a 795 nm photon. The branching ratio of the optical pumping to the dark state $\vert 2, -2 \rangle$ is $1/3$, which results in the scattering of an average of $3\sigma^-$ photons to optically pump the atoms following a Raman transition. In the limit of low scattering rate ($\Gamma_{sc} \ll \omega_{x,y,z}$ , which is required to limit reabsorption heating, see below), only the recoil energies of the scattered $\sigma^-$ photons add, on top of those of the spontaneously emitted photons, resulting in an average net recoil heating of $6E_r$ per optical pumping cycle. The average net energy removed per cooling cycle is then given by:
\begin{align}
    \Delta K_{total}&=\Delta K_{Raman} - 6E_r \\
    &= \hbar \delta_{\mathbf{v}} - 8 E_r
\end{align}
Therefore there is net cooling if and only if the two-photon Doppler shift is larger than $8E_r$, which translates to a kinetic energy in the $\Delta \mathbf{k}$ direction of:
\begin{equation}
    K_{\Delta \mathbf{k}} \geq 8 E_r = h \times 29 kHz
\end{equation}
As a result, for temperatures below $T^{eff}_r = 2.8 \mu K$ ($\langle K_{\Delta \mathbf{k}} \rangle = \frac{1}{2} k_B T^{eff}_r$ along $\Delta \mathbf{k}$, the cooling speed drops since only a small fraction of the atoms have Doppler shifts above 8Er where they can be cooled.

\subsection{OPTIMIZATION OF THE COOLING AND LIMITING FACTORS}
For each stage the Raman transitions are tuned to a particular velocity class by adjusting the magnetic field, which differentially shifts the $\vert 2, -2 \rangle$ and $\vert 2, -1 \rangle$ states. In stage S1, where the mean Doppler shift is much larger than the $8E_r$ limit, we find as expected that the optimal Raman detuning is near the rms Doppler shift, see Fig. S3(a). This is a compromise between removing a large amount of energy per cooling cycle ($\Delta K_{total} \approx \hbar \delta_R$) and having a sufficiently large probability of finding atoms at those velocities (which drops as $\delta_R$ increases). In stage X3 the optimization yields an optimal resonant Doppler shift of $\delta_R/(2 \pi) = 29$kHz. As shown in Fig. S3(b), the probability of finding an atom at these velocities is very small, and the amount of energy removed $\Delta K_{total}$ also becomes very small. By averaging $\Delta K_{Raman}$ over the actual distribution of addressed atoms [green shaded in Fig. S3(b)] in stage X3, we obtain $\langle \Delta K_{Raman} \rangle / h = 15$kHz, well below the recoil heating of $6E_r/h = 22$kHz, which should lead to heating. Yet cooling is observed, which suggests bosonic enhancement of the branching ratio into the $\vert2, -2\rangle$ state due to the emerging condensate. A better branching ratio would reduce the average recoil heating during optical pumping, and so cooling could be achieved even when the Raman transition removes less than $6E_r$ of kinetic energy.

At each stage, the strength of the Raman coupling $\Omega_R$ is optimized by scanning the power of the $\pi$ beam. Too small $\Omega_R$ lead to a narrow excitation profile, and therefore only a small fraction of the atoms undergo a Raman transition. However, if $\Omega_R$ is too large, already "cold" atoms undergo a Raman transition, due to the broadened excitation profile, and are heated during the optical pumping.

Initially in the sODT it is favorable to have a fairly large scattering rate for cooling speed. The smaller number of atoms available to cool in the Boltzmann tail above the recoil energy in the later stages results in a lower optimal scattering rate as it becomes more favorable to decrease reabsorption heating as seen in Fig. 4(c) and discussed below.

For the ODT powers, the main considerations are finding a balance between low density in order to limit inelastic loss and heating, and maintaining large enough trapping frequencies and therefore critical temperatures. Additionally in the final stage, having a low trap depth has proven crucial to avoid an observed density-dependent heating that increases with trap depth. The products of three-body recombination of $^{87}Rb$ atoms in the $\vert 2, -2 \rangle$ state to the least-bound molecular state ($h \times 24$MHz of binding energy) can collide with the cold sample before they leave the trap. This was shown to lead to large loss for a collisionally opaque ensemble. In our case, the sample is not collisionally opaque (collision probability $\sim 0.1$ for $s$-wave scattering only), nor is the trap deep enough to directly hold recombination products which would dump their energy into the cloud. The presence of a $d$-wave shape resonance at the energy of the least-bound state for a $^{87}Rb_2$ molecule is expected to enhance the collisional cross-section of recombination products with the trapped atoms, which could lead to strong losses in the collisionally thick regime. However, we mostly observe heating, which we suspect arises from recombination products undergoing grazing collisions with trapped atoms, with the latter remaining trapped and depositing heat into the cloud. We found that lowering the trap power as much as possible during the final stage X3 to minimize heating was necessary to reach condensation.

The optimized values of the relevant parameters throughout the sequence, namely trap power, Raman detuning $\delta_R$, Raman coupling $\Omega_R$ and scattering rate $\Gamma_{sc}$, are shown in Fig. S2 for reference.

\subsubsection{Description of Fig. S3}
Thermal velocity distribution (dashed purple), time averaged Lorentzian Raman excitation profile expected given the Raman Rabi frequency (solid blue line), rescaled product of the thermal distribution and the excitation profile (shaded green), and effective recoil limit of $29$kHz (dotted red) for cooling sequence parameters in (a) stage S1 and (b) stage X3. On average, a cooling cycle will cool atoms with velocity greater than the effective recoil limit, and heat atoms with velocity below the effective recoil limit.

\subsection{ESTIMATION OF THE REABSORPTION PROBABILITY}
The reabsorption of a scattered photon causes excess recoil heating. This is especially of concern since the reabsorption cross-section, corresponding to a two-photon resonant process, can take on its maximum possible value $\sim 6\pi \lambdabar^2$. Several strategies have been laid out for suppressing this, boiling down to the use of a low scattering rate $\Gamma_{sc}$ for optical pumping, a regime known as festina lente. With our optimal parameters, we have $\Gamma_sc \ll \omega_D, \omega_{x,y,z}$, leading to a suppression of the reabsorption process on the order of $\Gamma_{sc}/\omega_D$, where $\omega_D$ represents the Doppler width. The reabsorption cross-section $\sigma_{reabs}$ is given:
\begin{equation}
    \sigma_{reabs} \approx 4\pi \lambdabar^2 \frac{\sqrt{\pi} \Gamma_{sc}}{2 \sqrt{2} \omega_D}
\end{equation}
and the reabsorption probabiltiy $p_reabs$ is giveen by:
\begin{equation}
    p_{reabs} \approx \langle nl \rangle
\end{equation}
where $\langle nl \rangle$ is the mean column density of the cloud. We use the following forumla for the mean column density of a themal cloud,
\begin{equation}
    \langle nl \rangle = \sqrt{\frac{\pi}{8}}n(0) \sigma_z \frac{artanh (\sqrt{1-1/\epsilon^2)}}{\sqrt{\epsilon^2 - 1}}
\end{equation}
where $n(0)$ is the peak density, $\sigma_z$ is the cloud's waist along the short axis of an elongated trap ($z$-axis here), and $\epsilon = \omega_y / \omega_z$ is the aspect ratio of the trap. Assuming a classical cloud at the condensation point and a scattering rate $\Gamma_{sc} = 0.59 ms^{-1}$, we obtain that a probability for reabsorption $p_{reabs} \sim 0.1$. Hence reabsorption is not expected to be a significant factor, but it could explain why the performance deteriorates at larger $\Gamma_{sc}$ in Fig. 9(c) of the main text, since $p_{reabs}$ varies linearly with $\Gamma_{sc}$.

\end{document}
